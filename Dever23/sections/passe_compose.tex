\section{PASSÉ COMPOSÉ ET VERBES PRONOMINAUX}

\subsection{Mettez les verbes entre parenthèses au Passé Composé}

\begin{enumerate}
    \item Pierre s'est réveillé et s'est levé.
    \item Dans la salle de bains, il s'est douché et s'est lavé les cheveux.
    \item Ensuite, il s'est rasé, s'est brossé les dents et s'est coiffé.
    \item Puis, il s'est habillé et s'est approché de la table de la cuisine
    \item Enfin, il s'est installé pour prendre son petit déjeuner.
    \item Il s'est servi du café et s'est demandé comment serait la journée.
    \item Il s'est rappellé d'un rendez-vous chez le dentiste ce matin même
    \item Alors, il s'est dit que ça commençait très mal!
\end{enumerate}

\subsection{Racontez la même histoire à la première personne du singulier et du pluriel.}

\begin{enumerate}
    \item Je me suis réveillé et je me suis levé.
    \item Dans la salle de bains, je me suis douché et je me suis lavé les cheveux.
    \item Ensuite, je me suis rasé, je me suis brossé les dents et je me suis coiffé.
    \item Puis, je me suis habillé et je me suis approché de la table de la cuisine
    \item Enfin, je me suis installé pour prendre son petit déjeuner.
    \item Je me suis servi du café et je me suis demandé comment serait la journée.
    \item Je me suis rappellé d'un rendez-vous chez le dentiste ce matin même
    \item Alors, je me suis dit que ça commençait très mal!
\end{enumerate}

\begin{enumerate}
    \item Nous nous sommes réveillé et nous nous sommes levé.
    \item Dans la salle de bains, nous nous sommes douché et nous nous sommes lavé les cheveux.
    \item Ensuite, nous nous sommes rasé, nous nous sommes brossé les dents et nous nous sommes coiffé.
    \item Puis, nous nous sommes habillé et nous nous sommes approché de la table de la cuisine
    \item Enfin, nous nous sommes installé pour prendre son petit déjeuner.
    \item Nous nous sommes servi du café et nous nous sommes demandé comment serait la journée.
    \item Nous nous sommes rappellé d'un rendez-vous chez le dentiste ce matin même
    \item Alors, nous nous sommes dit que ça commençait très mal!
\end{enumerate}

\subsection{Mettez les verbes au passé changeant le pronom complément selon la personne}

\begin{enumerate}
    \item Je me suis senti mal, et je me suis couché.
    \item Tu t'es regardé au miroir, et tu t'es trouvé belle.
    \item Elle s'est entrainé beaucoup, alors elle s'est fatigué.
    \item Nous nous sommes promené dans le parc, et ensuite nous nous sommes assoi sur un banc.
    \item Vous vous êtes inquieté de votre santé, alors vous vous êtes nourri mieux.
    \item Les enfants se sont amusé au jardin, ils se sont sali.
    \item Quand je me suis décidé à partir, je me suis souvenir de mes rêves de jeunesse.
    \item Vous vous êtes habitué à votre nouveau emploi parce que vous vous êtes ne pas se découragé.
\end{enumerate}

\subsection{Attention à la negation avec les verbes pronominaux}

\begin{enumerate}
    \item Nous nous sommes promené dans la forêt, mais nous ne nous sommes pas interessé aux arbres.
    \item Quand il s'est perdré dans la ville, il ne s'est pas informé sur la bonne direction à prendre.
    \item Debout, l'accusé s'est défendré avec courage et il ne s'est pas asseoi.
    \item Julia s'est installé à sa place et ne s'est pas excusé.
    \item Je ne me suis pas se souveni si elle s'est maquillé.
\end{enumerate}

\subsection{Raconter l'histoire au passé}

Un jour, Alain et Anne sont se rencontré pour la première fois, à la piscine. D'abord, il s'est regardé, 
ensuite ils se sont parlé. Puis ils se sont souhaité une bonne soirée. Ils se sont tendré la main et ils se sont quitté.
Le lendemain, ils se sont téléphoné et se sont revoi. Pendant quelque temps, ils se sont entendré bien, mais un jour, ils
se sont disputé et se sont séparé. Pourtant, ils ne se sont pas oublié et ils se sont communiqué par des longs e-mails.
Ainsi, ils se sont connaîtré mieux. Deux ans après, ils se sont marié.