\section{Quel est le nom de l'objet}

\begin{enumerate}
    \item Une calculatrice.
    \item Un ordinateur.
    \item Une cassette vidéo.
    \item Un livre.
    \item Une chaise.
\end{enumerate}

\section{Associez les mots}

\begin{enumerate}
    \item Manger \rar faim.
    \item Chercer \rar  du travail.
    \item travailler \rar au restaurant.
    \item Avoir \rar un appartement.
    \item Présenter \rar son père.
    \item Visiter \rar le samedi.
\end{enumerate}

\section{Quel est le genre des noms?}

\begin{table}[H]
    \centering
    \begin{tabular}{|c|c|c|}
        \hline
        \textbf{Masculin} & \textbf{Féminin} & \textbf{Masculin ou féminin} \\
        \hline
        un cuisiner & une agence & un ou une stagiaire \\
        un garçon & une fille & un ou une locataire \\ 
        un prénom & une femme & un ou une secrétaire \\
        un directeur & une profession & un ou une dentiste \\
        un chanteurs & une adresse & \\
        & une actrice & \\
        \hline
    \end{tabular}
\end{table}

\section{Article défini ou indéfini?}

\begin{enumerate}
    \item Pascal, c'est \und{un} ami de Benoît et de Julie?
    \item Maintenant, oui. C'est \und{un} nouveau locataire.
    \item Il a \und{une} chambre?
    \item Oui. C'est \und{une} grande chambre.
    \item Et \und{un} bureau, \und{une} chaise et \und{un} lit.
\end{enumerate}

\section{Quel est le numéro de la photo?}

\begin{enumerate}
    \item C'est un beau musée. C'est le musée de Louvre.
    \item C'est une grande église. C'est le Sacré-Coeur de MontMartre.
    \item C'est un café célèbre. C'est le café de Flore à Saint-Germain-des-Prés.
    \item C'est une place de Paris, la place de la Condorde.
\end{enumerate}

\section{Qu'est-ce qu'il/elle a?}

\begin{enumerate}
    \item Elle aussi, il a un livre.
    \item Lui aussi, il a un appartement.
    \item Elle aussi, Elle a une grande cuisine.
    \item Moi aussi, j'ai un nouveau bureau.
    \item Moi aussi, j'ai faim.
\end{enumerate}

\section{Être ou avoir?}

\begin{enumerate}
    \item \und{Elle est} Marisa Ricci?
    \item Oui, \und{c'est} elle.
    \item Elle \und{est} française?
    \item Non, elle \und{est} italienne.
    \item Elle \und{a} quel âge?
    \item Elle \und{a} 25 ans.
    \item Elle \und{a} un appartement à Paris?
    \item Oui. Il \und{a} dans le $18^e$ arrondissement.
\end{enumerate}

\section{Expressions avec avoir.}

\begin{enumerate}
    \item Il est male de tête.
    \item Il est chaud.
    \item Elle est faim.
    \item Il est froid.
\end{enumerate}