\section{Présentez et dénisissez}

\begin{enumerate}
    \item Benoît travaille dans \und{une} agence. C'est \und{une} agence Europe Voyages.
    \item Voilà \und{un} chien. Ah, c'est \und{le} chien de la jeune femme.
    \item P.-H. de Latour est dans \und{un} appartement. C'est \und{l'}appartement de Julie et Benoît.
    \item Je vous présente Pascal. C'est \und{le} nouveau locataire.
    \item Voilà \und{une} table à repasser. C'est \und{la} table à repasser de Pascal.
\end{enumerate}

\section{Il manque les articles!}

\begin{enumerate}
    \item Benoît est \und{un} garçon courageux. Il travaille aussi \und{le} samedi.
    \item Il a \und{un} profession intéressante dans \und{l'}agence de voyages.
    \item \und{La} mère de Julie visite \und{l'}appartement.
    \item Voilà \und{une} salon, \und{une} cuisine et \und{une} salle de bains.
    \item Il y a \und{un} bon restaurant dans \und{le} quartier?
\end{enumerate}

\section{Quelle heure est-il?}

\begin{enumerate}
    \item Il est sept heurs matin.
    \item Il est onze heurs matin.
    \item Il est deux heurs après midi.
\end{enumerate}

\section{Être ou avoir?}

\begin{enumerate}
    \item Vous \und{avez} une profession intéressante? \\ 
          Oui, je \und{suis} journaliste.
    \item Tu \und{es} étudiante? Oui, je \und{suis} étudiante.
    \item Elle \und{a} un grand appartement. \\
          Oui, il \und{a} grand.
    \item Vous \und{avez} soif? Oui j'\und{ai} soif et j'\und{ai} faim. 
\end{enumerate}