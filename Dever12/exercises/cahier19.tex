
\section{Pluirel des verbs}

\begin{enumerate}
    \item êtes
    \item sont
    \item avez
    \item sont
    \item sommes
    \item ont
    \item sont
    \item avon
    \item sont
    \item ont
\end{enumerate}

\section{Ils sont plusieurs! Mettez au pluriel.}

\begin{enumerate}
    \item Ils habitent à Bordeaux.
    \item Vous offrez des fleurs.
    \item Nous travaillons à l'agence de voyages.
    \item Elles aiment des fleurs.
    \item Ils mangent des gâteux.
\end{enumerate}

\section{Mettez au pluriel.}

Ce sont jeune filles sympathiques. Elles travaillent dans une agence de voyages.
Elles plaisantent avec les collègue de bureau. Elles tutoie la responsable du service.
Elles ont toujours des bonnes idées: offrir les beau bouquet pour l'anniversaire, faire
un bon gâteau, faire les plaisanteries gentilles. Ils aiment bien les jeune filles ausi
aimables et séuriuses.

% \section{Genre et place des adjectifs.}

% \begin{enumerate}
%     \item C'est une \und{bonne} idée.
%     \item C'est un \und{bel} cadeau.
%     \item C'est une collègue \und{sympathique}.
%     \item C'est un \und{grand} appartement.
%     \item C'est un \und{bel} copain.
%     \item C'est un garçon \und{sérieux}.
%     \item C'est une \und{nouveau} stagieire.
%     \item C'est un \und{petit} restaurant.
%     \item C'est une cliente \und{difficile}.
% \end{enumerate}