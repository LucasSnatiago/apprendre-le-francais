\section{C'est quand?}

\begin{enumerate}
    \item L'anniversaire de Benoît est le 5 avril.
    \item Quand vous êtes un bon ami.
    \item Mon anniversaire est le seize juin.
    \item C'est mardi.
    \item C'est janvier.
\end{enumerate}

\section{Masculin, féminin}

\begin{enumerate}
    \item C'est un fils gentil.
    \item C'est un jeune fils sérieux.
    \item Il a un bon ami.
    \item Nous avons un nouveau stagiaire.
    \item Il a un bel chien.
\end{enumerate}

\section{Négation et pronoms tonique au pluriel.}

\begin{enumerate}
    \item Non, elles, elles n'offrent pas des fleurs.
    \item Non, eux, ils ne mangent pas des gâteaux le dimanche.
    \item Non, eux, ils n'ont pas des amis.
    \item Non, elles, elles ne travaillent pas dans des grande agence.
    \item Non, vous, vous n'achetez pas de billet.
    \item Non, nous, nous ne passons pas de banque le jeudi.
\end{enumerate}

\section{Forme négative de l'imperatif.}

\begin{enumerate}
    \item Non, ne payez pas par chèque.
    \item Non, ne plaisantentez pas avec les clients.
    \item Non, ne pose pas les lettres sur le bureau.
    \item Non, n'emmène pas le stagiaire dans le bureau.
    \item Non, n'achète pas des fleurs pour Benoît.
\end{enumerate}