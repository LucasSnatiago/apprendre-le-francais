\section{Cahier d'exercices}

\subsection{Pronoms personnels et toniques}
\begin{enumerate}
    \item Salut. Tu t'appelles comment?
    \item Éric. Et tu t'appelles comment?
    \item Moi. Je m'appelle Sophie.
    \item Tu es actrice.
    \item No. Je suis étudiante.
    \item Et il?
    \item Lui, il est professeur.
\end{enumerate}

\subsection{être et s'appeler}
\begin{enumerate}
    \item Bonjour. Vous vous appellez comment?
    \item Joseph Pinson. Et vous?
    \item Moi, je m'appelle Valérie Moreau.
    \item Vous être journaliste?
    \item Oui. Et vous, vous être acteur?
    \item Oui. Je suis acteur.
    \item Et elle? Elle est actrice?
    \item Oui, elle est actrice.
\end{enumerate}

\subsection{Le dialogue}
\begin{enumerate}[label=\alph*.]
    \item[b.] Moi, c'est Alberto.
    \item[d.] C'est Paolo, un ami.
    \item[c.] Et lui. c'est qui?
    \item[f.] Non, je suis italien.
    \item[e.] Tu es espagnol?
    \item[a.] Salut, je m'appelle Quentin. Et toi?
\end{enumerate}

\subsection{Un ou une?}
\begin{enumerate}
    \item C'est un \textbf{français} \emph{française}.
    \item C'est une \emph{étudiant} \textbf{étudiante}.
    \item C'est un \emph{actrice} \textbf{acteur}.
    \item C'est une \emph{ami} \textbf{amie}.
    \item C'est \textbf{italien} \emph{italienne}.
\end{enumerate}