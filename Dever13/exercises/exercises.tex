\section{Soulignez l'adjectif correct.}

\begin{enumerate}
    \item australien
    \item camerounaise
    \item égyptienne
    \item suédoise
    \item espagnole
    \item chinois
    \item mexicain
    \item argentine
\end{enumerate}

\section{Soulignez le sujet correct.}

\begin{enumerate}
    \item Ce sac
    \item Cette bouteille
    \item Ce vase
    \item Le coffre est ouvert
    \item L'eau 
    \item Le salon
    \item Cette maison
    \item Cette exposition
\end{enumerate}

\section{Soulignez l'adjectif}

\begin{enumerate}
    \item jolie / grande
    \item petit / rond / frisé
    \item âgé / musclé
    \item brune / blonde
    \item mignon
    \item gros
\end{enumerate}

\section{Qui parle? Un homme (H), une femme (f), un homme ou une femme (h/f)}

\begin{enumerate}
    \item H
    \item F
    \item F
    \item F
    \item h/f
    \item F
    \item H
    \item H
    \item H
    \item h/f
\end{enumerate}

\section{Trouvez les adjectifs masculins pour Michel et féminin pour Carole.}

Michel est:
\begin{enumerate}
    \item beau 
    \item sportif
    \item turc
    \item fou
    \item menteur
    \item vieux
    \item sérieux
    \item roux
    \item gentil
    \item doux
\end{enumerate}

Carole est:
\begin{enumerate}
    \item belle
    \item sportive
    \item folle
    \item sérieuse
    \item vieille
    \item rousse
    \item douce 
    \item turque
    \item menteuse
    \item gentille
\end{enumerate}

\section{Complétez}

\begin{enumerate}
    \item Ma cousine est très intelligente mais un peu folle.
    \item Mon père est naturel et doux.
    \item Ma soeur est menteuse et impolie.
    \item Mon grand-père est un homme passionné et cultivé.
    \item Ma tante est très jalouse et parfois violente.
    \item Mon beau-père est généreux et gentil.
\end{enumerate}

\section{Complétez avec les adjectifs jaune, vert, blanc, rouge, noir.}

\begin{enumerate}
    \item blanc / jaune
    \item rouge / vert / jaune
    \item noir, blanc
    \item vert, noir
\end{enumerate}

\section{Soulignez l'adijectif correct.}

\begin{enumerate}
    \item faux
    \item longue
    \item doux
    \item frais 
    \item sec 
    \item neuve
    \item léger 
    \item gros
\end{enumerate}

\section{Trouvez un contraire}

\begin{enumerate}
    \item 
\end{enumerate}

\section{Mettez au féminin si nécessaire et dites si la personne est appréciée ou pas.}

\begin{enumerate}
    \item lente / Non 
    \item fantastique / Oui
    \item bruyante / Non
    \item intéressant / Oui
    \item spontanée / Oui
    \item malhonnête / Non
    \item gentille / Oui
    \item bavarde / Non
    \item génialle / Oui 
    \item indiscret / Non
\end{enumerate}

\section{Qui parle? Patrick et Marie? Marie et Valérie? Patrick et Arthur?}

\begin{tabular}{|c|c|c|}
    \hline
    Patrick et Marie & Marie et Valérie & Patrick et Arthur \\
    \hline
    & X & \\
    \hline
    & X & \\
    \hline
    & & X \\
    \hline
    X & & \\
    \hline
    & & X \\
    \hline
    X & & \\
    \hline
    & & X \\
    \hline
    X & & \\
    \hline
    & & X \\
    \hline
    & X & \\
    \hline
\end{tabular}

\section{Indiquez si l'adjectif est au singulier (S), au pluriel (P) ou peut être singulier ou pluriel (S/P) puis complétez la deuxième partie.}

\begin{enumerate}
    \item P
    \item S
    \item S
    \item S/P
    \item P
    \item S/P
    \item S
    \item P
    \item S
    \item S
\end{enumerate}

Ce livre est gros, épais, lourd, faux, cher, ancien, unique.

Ces livres sont intéressants, épais, précieux, faux, nouveaux.

\section{Indiquez si l'adjectif est au singulier (S), au pluriel (P) ou peut être singulier ou pluriel (S/P) puis complétez la deuxième partie.}

\begin{enumerate}
    \item S/P
    \item S
    \item P
    \item P
    \item S
    \item S/P
    \item S
    \item P
\end{enumerate}

Cet homme est vieux, égoïste, maigre, roux, menteur.
Ces hommes sont vieux, beaux, joyeux, roux, naturels.

\section{Mettez les adijectif au pluriel.}

\begin{enumerate}
    \item locaux.
    \item régionaux.
    \item spéciales.
    \item nouveaux.
    \item internationaux.
    \item tranquilles.
\end{enumerate}

\section{Faites comme dans l'exemple.}

\begin{enumerate}
    \item longues, complexes, passionnantes.
    \item populaires, animées, commerçantes. *
    \item économiques, régionaux *
    \item anciens, merveilleux, froids, inconfortables.
    \item belles, rares, cheres.
\end{enumerate}