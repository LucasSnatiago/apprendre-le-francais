\section{Quel pronom personnel?}

\begin{enumerate}
    \item \textbf{Tu} es secrétaire?
    \item \textbf{Vous} êtes fraiçaise?
    \item \textbf{Je} suis étudiant.
    \item \textbf{Il} est agent de voyages.
    \item \textbf{Elle} est étudiante.
    \item \textbf{Tu} es stagiaire.
    \item \textbf{Je} suis allemande.
    \item \textbf{Elle} est mannequin.
\end{enumerate}

\section{Qui est-ce?}

\begin{enumerate}
    \item C'est toi, Pascal? \rar Oui, c'est moi.
    \item Benôit, c'est lui? \rar Oui, c'est lui.
    \item Qui est-ce? C'est Julie? \rar Oui, c'est elle.
    \item Qui est-ce? C'est Pascal? \rar Oui, c'est il.
    \item Julie, c'est vous? \rar Oui, c'est moi.
    \item C'est chez toi, ici? \rar Oui, c'est chez moi, ici.
\end{enumerate}

\section{Présentations.}

\begin{enumerate}
    \item Moi, je \underline{suis} étudiant(e). Et toi, tu \underline{es} étudiant(e).
    \item Non, moi, je \underline{suis} agent de voyages.
    \item Et Erica, elle \underline{est} allemande?
    \item Oui, elle \underline{est} allemande.
    \item Et elle \underline{est} étudiante?
    \item Non, elle \underline{est} mannequin. 
\end{enumerate}
