\section{Chassez l'intrus.}

\begin{enumerate}
    \item Un dentiste - un médicin - l'âge - \sout{la vidéo}.
    \item Un mannequin - un médicin - \sout{un chein} - une secrétaire.
    \item Un mannequin - un agent de voyages - un locataire - \sout{un étudiant}.
    \item Moi - toi - \sout{je} - lui.
\end{enumerate}

\section{Quelle est leur profession?}

\begin{enumerate}
    \item Ingrid est \underline{mannequin}.
    \item Catherine Deneuve est \underline{actrice}.
    \item Pierre-Henri de Latour est \underline{}.
\end{enumerate}

\section{Quelle est leur identité?}

\section{Homme ou femme?}

\begin{enumerate}
    \item \textbf{H} \rar Is est agent de voyages?
    \item \textbf{H} \rar C'est un garçon heureux
    \item \textbf{F} \rar Dominique est française?
    \item \textbf{F} \rar Je vous présente le nouveau locataire.
    \item \textbf{F} \rar Ton amie Valérie va bien?
\end{enumerate}

\section{Le verbe être.}

\begin{enumerate}
    \item Tu \underline{es} fraiçaise?
    \item Vous \underline{êtes} italien?
    \item Il \underline{est} dans sa chambre.
    \item Je \underline{suis} espagnol. 
    \item Vous \underline{êtes} seul?
    \item Tu \underline{es} dans le salon?
    \item Elle \underline{est} italienne.
\end{enumerate}

\section{Les pronoms toniques.}

\begin{enumerate}
    \item \underline{Moi}, je suis secrétaire. Et \underline{lui}? \underline{Lui} aussi.
    \item J'habite à Paris. Et \underline{vous}? - \underline{Moi}, j'habite à Lyon.
    \item Et \underline{lui}, il s'apelle comment? \underline{Lui}, c'est Alain.
    \item \underline{Elle}, elle est étudiante. Et \underline{tu}? - \underline{Moi}, je suis dentiste.
    \item Je m'apelle Legrand. Et \underline{vous}? - je m'apelle Berthier.
\end{enumerate}

\section{Mettez ensemble questions et réponses.}

\begin{enumerate}
    \item Bonjour, Monsieur. \rar D
    \item Vous vous appellez comment? \rar A
    \item Quelle est votre adresse? \rar B
    \item Non. Excusez-moi. Votre adresse à Paris, pas à Barcelone. \rar E
    \item Vous avez un numéro de téléphone? \rar C
\end{enumerate}

\begin{enumerate}[label=\alph*)]
    \item Edmundo Rojas.
    \item À Barcelone?
    \item Oui, C'est le 01 44 37  18 76.
    \item Bounjour.
    \item Ah, bon. J'habite chez un ami, au Quartier latin, 18 rue Hautefeuille.
\end{enumerate}

\section{Prépositions.}

\begin{enumerate}
    \item 
\end{enumerate}